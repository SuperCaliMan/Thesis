\begin{table}
	\caption{Tabella requisiti}
	\begin{tabular}{||l||l|l|}
		\hline
		\# & Descrizione & Priorità \\ \hline
		1 & Deve presentare una schermata di login,  l'utente può autenticarsi con nome utente e password. & Alta \\ \hline
		2 & Inserire l'orario di lavoro, impostando: ora ingresso e uscita & Alta \\ \hline
		3 & Visualizzare il mese corrente, composto da 6 settimana e con il giorno attuale selezionato automaticamente & Alta \\ \hline
		4 & Inserire una nuova commessa impostando i seguenti parametri obbligatori: nome commessa, riga, operazione, ore e note & Alta \\ \hline
		5 & Inserire i dati per il rimborso delle spese: vitto, altre spese, alloggio, km, fascia chilometrica & Alta \\ \hline
		6 & Impostare le assenze e gli straordinari in un singolo giorno & Alta \\ \hline
		7 & Eseguire il logout cancellando le credenziali se salvate in fase di login & Alta \\ \hline
		8 & Cancellazione degli elementi nelle liste tramite swipe e pressione prolungata & Alta \\ \hline
		9 & Il calendario, deve presentare in ogni giorno due barre che segnalano se c'è una timbratura per quel giorno e se il numero di ore inserite tramite le commesse coincide con quelle timbrate & Alta \\ \hline
		10 & Visualizzare sei settimane nel calendario e rendere selezionabili solo i giorni del mese corrente & Alta \\ \hline
		11 & Salvare le commesse in un database locale. Gli ordini salvati saranno le commesse preferite dell'utente, le quali verrano visualizzate in fase di inserimento & Media \\ \hline
		12 & Cancellare le commesse salvate & Media \\ \hline
		13 & Visualizzare il mese precedente con tutti i dati relativi ai giorni & Media \\ \hline
		14 & Cancellare tutti i dati inseriti in un singolo giorno & Media \\ \hline
		15 & Visualizzare l'utente attualmente loggato con la relativa email & Media \\ \hline
		16 & Ricercare una commessa tramite: codice, descrizione o nome azienda & Media \\ \hline
		17 & Visualizzare la lista di tutte le commesse ordinata per data di inserimento & Media \\ \hline
		18 & Ricevere una notifica ogni inizio mese che ricorda di chiudere tutti i giorni del mese & Bassa \\ \hline
		19 & Sezione privacy & Bassa \\ \hline
		20 & Inserire altre coppie di valori (ingresso, uscita) per un massimo di 10 e un minimo di 2 & Bassa \\ \hline
		21 & Disattivare le notifiche & Bassa \\ \hline
		22 & Inserire tramite una sezione apposita, consigli e suggerimenti & Bassa \\ \hline
	\end{tabular}
\end{table}